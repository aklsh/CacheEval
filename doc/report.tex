\documentclass[12pt,a4paper,english]{paper} 
\usepackage{fontspec}
\usepackage[utf8]{inputenc}
\usepackage[margin=2cm]{geometry}
\usepackage{multirow}
\usepackage{hyperref}
\usepackage{graphicx}
\usepackage{bm}
\usepackage[usenames,dvipsnames]{color}
\usepackage{booktabs}
\usepackage{fancyhdr}
\usepackage[most]{tcolorbox}
\usepackage{changepage}
\usepackage[square,sort,comma,numbers]{natbib}
\usepackage{amsmath}
\usepackage{amssymb}
\usepackage{eucal}
\usepackage[]{minted}
\usepackage{latexsym}
\usepackage{indentfirst}
\usepackage[ruled,vlined]{algorithm2e}
\usepackage[english]{babel}
\usepackage[autostyle, english = american]{csquotes}
\MakeOuterQuote{"}

\setmonofont[Scale=0.75]{Hack NF}

\def \courseNumber {CS6600}
\def \courseName {Computer Architecture}
\def \assignmentName {Reverse Engineering L1 Cache}
\def \myName {Akilesh Kannan}
\def \rollNumber {EE18B122}

\setlength{\headheight}{14pt}

\pagestyle{fancy}
\fancyhf{}
\rhead{\assignmentName}
\lhead{\courseNumber: \courseName}
\cfoot{\thepage}

% \linespread{1.2}

\renewcommand{\familydefault}{\sfdefault} %command to change font to sans-serif

\definecolor{blue(ryb)}{rgb}{0.01, 0.28, 1.0}
\definecolor{green(ryb)}{rgb}{0.28, 1.0, 0.01}
\definecolor{red(ryb)}{rgb}{1.0, 0.01, 0.28}
\definecolor{black(ryb)}{rgb}{0, 0, 0}
\definecolor{gray(ryb)}{rgb}{0.75, 0.75, 0.75}
\definecolor{orange}{RGB}{255,155,0}
\definecolor{formalblue}{rgb}{0.95,0.95,1}
\definecolor{formalred}{rgb}{1,0.95,0.95}

\newenvironment{colorboxed}[4][gray]{
\begin{tcolorbox}[colback=#1!3!white,colframe=#1(ryb)!50!black,title=\textbf{#2 #3},#4]
}{
\end{tcolorbox}
}

\newenvironment{warning}{%
  \def\FrameCommand{%
    \hspace{1pt}%
    {\color{red}\vrule width 2pt}%
    {\color{formalred}\vrule width 4pt}%
    \colorbox{formalred}%
  }%
  \MakeFramed{\advance\hsize-\width\FrameRestore}%
  \noindent\hspace{-4.55pt}% disable indenting first paragraph
  \begin{adjustwidth}{7pt}{}%
  \vspace{2pt}\vspace{2pt}%
}
{%
  \vspace{2pt}\end{adjustwidth}\endMakeFramed%
}

\newenvironment{results}{%
  \def\FrameCommand{%
    \hspace{1pt}%
    {\color{blue}\vrule width 2pt}%
    {\color{formalblue}\vrule width 4pt}%
    \colorbox{formalblue}%
  }%
  \MakeFramed{\advance\hsize-\width\FrameRestore}%
  \noindent\hspace{-4.55pt}% disable indenting first paragraph
  \begin{adjustwidth}{7pt}{}%
  \vspace{2pt}\vspace{2pt}%
}
{%
  \vspace{2pt}\end{adjustwidth}\endMakeFramed%
}

\begin{document} 
\thispagestyle{empty}
\vspace{-4.5cm}

\hspace*{-\parindent}
\begin{minipage}{0.65\textwidth}
\fontsize{22pt}{10pt}\selectfont\textbf{\assignmentName}\\[1mm]
\Large
\textit{\courseNumber: \courseName}\\[5mm]
\Large \myName \\
\normalsize \rollNumber \\
\end{minipage}\hfill% push everything to the right
\raisebox{-13mm}{\includegraphics[scale=.28]{logo.pdf}}

\hrule \hrule
\medskip


\begin{warning}
    This assignment was worked on individually.\\
    The programs and tools used can be found here - \url{https://github.com/aklsh/CacheEval}
\end{warning}

\section{System Details}
\begin{table}[H]
    \centering
    \resizebox{0.5\textwidth}{!}{%
    \begin{tabular}{|c|c|c|}
        \hline
         \textbf{Parameter} & \multicolumn{2}{c|}{\textbf{Value}} \\ \cline{2-3} 
            & \textit{Predicted} & \textit{Expected \footnotemark} \\ \hline
        {Associativity}        &         8          &        8            \\ \hline
        {Block Size}           &       64 Bytes     &     64 Bytes        \\ \hline
        {Processor Name}       &         -          &  AMD Ryzen 7 4800HS \\ \hline
        {ISA}                  &         -          &     x86\_64         \\ \hline
        {L1 D\$ size}          &         -          &      32 KB          \\ \hline
    \end{tabular}
    }
    \caption{Summary Table}
\end{table}
\footnotetext{On GNU/Linux OSes, it is obtained using \texttt{grep . /sys/devices/system/cpu/cpu0/cache/index*/*}}

\section{Approach}
\begin{results}
    \textsl{Misses in the cache lead to increased latency in memory access.}
\end{results}

\subsection{Measuring Block Size}
\begin{algorithm}[H]
\SetAlgoLined
\KwResult{Block Size}
 \SetKwInput{input}{Input}
 \input{stride values}
 \ForEach{stride}{
 \SetKw{init}{initialise} \init cacheBlock[64M], sum $=$ 0, startTime $=$ 0, endTime $=$ 0\; 
 \SetKw{clear}{clear} \clear cache\;
 \For{i $\leftarrow$ 1 \KwTo 64M}{
  startTime $\leftarrow$ RDTSC\;
    \SetKw{access}{access}\access cacheBlock[i]\;
  endTime $\leftarrow$ RDTSC\;
  \SetKw{print}{print}\print endTime $-$ startTime\;
  i $\leftarrow$ i $+$ stride\;
 }
}
    \SetKw{plot}{plot} \plot data\;
 \caption{Calculating Block Size of L1 D\$}
    \label{Alg1}
\end{algorithm}

The \texttt{RDTSC} instruction returns the data in the Time-Stamp Counter in x86 platforms \cite{intel_2021}.\\

Algorithm \ref{Alg1} is first tried with a stride of 1 byte. The latencies are then plotted to obtain the plots in Fig.\ref{Fig:blockSize}. From here, we can see that for every 64 bytes, there is a peak in the latency. This tells us that every 64$^{th}$ byte that is accessed, there is a miss in the cache and, the required cache block has to be retrieved from memory - giving us the cache block size.

Another way to confirm the same is by taking different strides across the array, and plotting the access latencies for each memory access. As shown in Fig.\ref{Fig:blockSize}, we can see that every 2$^{nd}$ 32 byte stride takes a high latency, every 4$^{th}$ 16 byte stride takes a high latency, and almost every 64 byte stride takes a high latency. Thus, we again get a block size of 64 bytes.

\begin{figure}[H]
    \centering
    \setlength\tabcolsep{1pt}
    \begin{tabular}{cc}
        \def\svgwidth{0.5\columnwidth}
        \input{blockSize_1B.pdf_tex}&
        \def\svgwidth{0.5\columnwidth}
        \input{blockSize.pdf_tex}
    \end{tabular}
    \caption{Calculating L1-D\$ block size}
    \label{Fig:blockSize}
\end{figure}

\subsection{Measuring Associativity}
On obtaining the block size using above technique, we can reverse-engineer the associativity of the cache by repeatedly accessing blocks that we expect to belong to the same set. On encountering a higher latency than previous blocks, we can conclude that all blocks in the same set have been filled, and thus there is a replacement that is done, resulting in the higher latency.

\begin{algorithm}[H]
\SetAlgoLined
\KwResult{Associativity}
 \KwIn{ASSOC, where ASSOC is the assumed value of associativity}
 \SetKw{init}{initialise} \init cacheBlock[64M], i $=$ 0, startTime $=$ 0, endTime $=$ 0\; 
 \SetKw{clear}{clear} \clear cache\;
    \While{i $<$ ASSOC}{
  startTime $\leftarrow$ RDTSC\;
    \SetKw{access}{access}\access cacheBlock[i $*$ CACHE\_SIZE]\;
  endTime $\leftarrow$ RDTSC\;
  \SetKw{print}{print}\print endTime $-$ startTime\;
  i $\leftarrow$ i $+$ 1\; 
    }
  startTime $\leftarrow$ RDTSC\;
    \SetKw{access}{access}\access cacheBlock[16 $*$ CACHE\_SIZE]\;
  endTime $\leftarrow$ RDTSC\;
  \SetKw{print}{print}\print endTime $-$ startTime\;
 \caption{Calculating Associativity of L1 D\$}
    \label{Alg2}
\end{algorithm}

The final access for \texttt{cacheBlock[16*CACHE\_SIZE]} is to see the latency on accessing \texttt{ASSOC}+1 blocks in the same set. 

Algorithm \ref{Alg2} was tried with the following \texttt{ASSOC} values - 2, 4, 8, 16, as most modern processors fall into one of these categories and the following outputs were got:

\begin{colorboxed}{Output: }{\texttt{\textbf{ASSOC = 2}}}{breakable}
\begin{verbatim}
---------START Populating---------
Byte 0: 	2146
Byte 32768: 	12789
---------FINISH Populating---------
---------START New Access---------
Byte 524288: 	5916
Byte 0: 	638
Byte 524288: 	174
Byte 0: 	174
Byte 524288: 	667
Byte 0: 	174
Byte 524288: 	203
Byte 0: 	174
---------FINISH New Access---------
\end{verbatim}
\end{colorboxed}

\begin{colorboxed}{Output: }{\texttt{\textbf{ASSOC = 4}}}{breakable}
\begin{verbatim}
---------START Populating---------
Byte 0: 	1479
Byte 32768: 	14819
Byte 65536: 	5655
Byte 98304: 	5539
---------FINISH Populating---------
---------START New Access---------
Byte 524288: 	6322
Byte 0: 	638
Byte 524288: 	203
Byte 0: 	174
Byte 524288: 	174
Byte 0: 	174
Byte 524288: 	174
Byte 0: 	174
---------FINISH New Access---------
\end{verbatim}
\end{colorboxed}


\begin{colorboxed}{Output: }{\texttt{\textbf{ASSOC = 8}}}{breakable}
\begin{verbatim}
---------START Populating---------
Byte 0: 	1247
Byte 32768: 	13688
Byte 65536: 	5481
Byte 98304: 	4466
Byte 131072: 	6177
Byte 163840: 	7569
Byte 196608: 	4872
Byte 229376: 	4756
---------FINISH Populating---------
---------START New Access---------
Byte 524288: 	20590
Byte 0: 	696
Byte 524288: 	174
Byte 0: 	174
Byte 524288: 	174
Byte 0: 	174
Byte 524288: 	174
Byte 0: 	203
---------FINISH New Access---------
\end{verbatim}
\end{colorboxed}

\begin{colorboxed}{Output: }{\texttt{\textbf{ASSOC = 16}}}{breakable}
\begin{verbatim}
---------START Populating---------
Byte 0: 	1334
Byte 32768: 	14268
Byte 65536: 	5887
Byte 98304: 	4785
Byte 131072: 	6525
Byte 163840: 	4988
Byte 196608: 	4263
Byte 229376: 	4901
Byte 262144: 	17371
Byte 294912: 	5336
Byte 327680: 	7424
Byte 360448: 	4669
Byte 393216: 	5075
Byte 425984: 	4727
Byte 458752: 	4640
Byte 491520: 	5684
---------FINISH Populating---------
---------START New Access---------
Byte 524288: 	4582
Byte 0: 	696
Byte 524288: 	203
Byte 0: 	174
Byte 524288: 	174
Byte 0: 	203
Byte 524288: 	203
Byte 0: 	203
---------FINISH New Access---------
\end{verbatim}
\end{colorboxed}

From the above results, we can see that for \texttt{ASSOC} of 2 or 4, a new access of \texttt{Byte 524288}, which belongs to block number 8192 has comparable latency to the previous accesses. But for \texttt{ASSOC} of 8, we see a huge increase in the time taken to access the same block. This can be inferred as the time taken to calculate the replacement candidate. We also see the same when we take \texttt{ASSOC} of 16 - the 8$^{th}$ block access has much higher latency than previous accesses.

Another insight from the output is that the evicted block was Block 0 - it was evicted from L1 to L2. Later upon accessing the same, we bring it back in the L1 cache, although at a slightly higher latency than L1 cache access.

%Beginning References. Don't add any text beyond this.
%------------------------------------------

\newpage %sending References to the last page

\bibliography{paper}
\bibliographystyle{acm}
\end{document}
