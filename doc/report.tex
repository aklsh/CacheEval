\documentclass[12pt,a4paper,english]{paper} 
\usepackage{fontspec}
\usepackage[utf8]{inputenc}
\usepackage[margin=2cm]{geometry}
\usepackage{multirow}
\usepackage{graphicx}
\usepackage{bm}
\usepackage[usenames,dvipsnames]{color}
\usepackage{booktabs}
\usepackage{fancyhdr}
\usepackage[most]{tcolorbox}
\usepackage{changepage}
\usepackage[square,sort,comma,numbers]{natbib}
\usepackage{amsmath}
\usepackage{amssymb}
\usepackage{eucal}
\usepackage[]{minted}
\usepackage{latexsym}
\usepackage{indentfirst}
\usepackage[ruled,vlined]{algorithm2e}
\usepackage[english]{babel}
\usepackage[autostyle, english = american]{csquotes}
\MakeOuterQuote{"}

\setmonofont[Scale=0.75]{Hack NF}

\def \courseNumber {CS6600}
\def \courseName {Computer Architecture}
\def \assignmentName {Reverse Engineering L1 Cache}
\def \myName {Akilesh Kannan}
\def \rollNumber {EE18B122}

\setlength{\headheight}{14pt}

\pagestyle{fancy}
\fancyhf{}
\rhead{\assignmentName}
\lhead{\courseNumber: \courseName}
\cfoot{\thepage}

% \linespread{1.2}

\renewcommand{\familydefault}{\sfdefault} %command to change font to sans-serif

\definecolor{blue(ryb)}{rgb}{0.01, 0.28, 1.0}
\definecolor{green(ryb)}{rgb}{0.28, 1.0, 0.01}
\definecolor{red(ryb)}{rgb}{1.0, 0.01, 0.28}
\definecolor{black(ryb)}{rgb}{0, 0, 0}
\definecolor{gray(ryb)}{rgb}{0.75, 0.75, 0.75}
\definecolor{orange}{RGB}{255,155,0}
\definecolor{formalblue}{rgb}{0.95,0.95,1}
\definecolor{formalred}{rgb}{1,0.95,0.95}

\newenvironment{colorboxed}[4][gray]{
\begin{tcolorbox}[colback=#1!3!white,colframe=#1(ryb)!50!black,title=\textbf{#2 #3},#4]
}{
\end{tcolorbox}
}

% environment derived from framed.sty: see leftbar environment definition

\newenvironment{formalred}{%
  \def\FrameCommand{%
    \hspace{1pt}%
    {\color{red}\vrule width 2pt}%
    {\color{formalred}\vrule width 4pt}%
    \colorbox{formalred}%
  }%
  \MakeFramed{\advance\hsize-\width\FrameRestore}%
  \noindent\hspace{-4.55pt}% disable indenting first paragraph
  \begin{adjustwidth}{7pt}{}%
  \vspace{2pt}\vspace{2pt}%
}
{%
  \vspace{2pt}\end{adjustwidth}\endMakeFramed%
}
\newenvironment{formalblue}{%
  \def\FrameCommand{%
    \hspace{1pt}%
    {\color{blue}\vrule width 2pt}%
    {\color{formalblue}\vrule width 4pt}%
    \colorbox{formalblue}%
  }%
  \MakeFramed{\advance\hsize-\width\FrameRestore}%
  \noindent\hspace{-4.55pt}% disable indenting first paragraph
  \begin{adjustwidth}{7pt}{}%
  \vspace{2pt}\vspace{2pt}%
}
{%
  \vspace{2pt}\end{adjustwidth}\endMakeFramed%
}

\begin{document} 
\thispagestyle{empty}
\vspace{-4.5cm}

\hspace*{-\parindent}
\begin{minipage}{0.65\textwidth}
\fontsize{22pt}{10pt}\selectfont\textbf{\assignmentName}\\[1mm]
\Large
\textit{\courseNumber: \courseName}\\[5mm]
\Large \myName \\
\normalsize \rollNumber \\
\end{minipage}\hfill% push everything to the right
% \raisebox{-13mm}{\includegraphics[scale=.28]{logo.pdf}}

\hrule \hrule
\medskip
% \vspace{1cm}


\begin{formalred}
    This assignment was worked on individually.
\end{formalred}

\section{System Details}
\begin{table}[H]
    \centering
    \resizebox{0.65\textwidth}{!}{%
    \begin{tabular}{|c|c|c|}
        \hline
         \textbf{Parameter} & \multicolumn{2}{c|}{\textbf{Value}} \\ \cline{2-3} 
            & \textit{Predicted} & \textit{Expected \footnotemark} \\ \hline
        {Associativity}        &                    &        8           \\ \hline
        {Block Size}           &       64 Bytes     &     64 Bytes       \\ \hline
        {Processor Name}       & AMD Ryzen 7 4800HS &     -               \\ \hline
        {ISA}                  & x86\_64            &     -               \\ \hline
    \end{tabular}
    }
    \caption{Summary Table}
\end{table}
\footnotetext{On GNU/Linux-based OSes, it is obtained using \texttt{grep . /sys/devices/system/cpu/cpu0/cache/index*/*}}

\section{Approach}
\begin{formalblue}
    \textsl{Misses in the cache lead to increased latency in memory access.}
\end{formalblue}

\subsection{Measuring Block Size}
\begin{algorithm}[H]
\SetAlgoLined
\KwResult{Block Size}
 \SetKwInput{input}{Input}
 \input{stride}
 \SetKw{init}{initialise} \init cacheBlock[64M], sum $=$ 0, startTime $=$ 0, endTime $=$ 0\; 
 \SetKw{clear}{clear} \clear cache\;
 \For{i $\leftarrow$ 1 \KwTo 64M}{
  startTime $\leftarrow$ RDTSC\;
  sum $\leftarrow$ sum $+$ cacheBlock[i]\;
  endTime $\leftarrow$ RDTSC\;
  \SetKw{print}{print}\print endTime $-$ startTime\;
  i $\leftarrow$ i $+$ stride\;
 }
 \caption{Calculating Block Size of L1 D\$}
\end{algorithm}

The \texttt{RDTSC} instruction returns the data in the Time-Stamp Counter in x86 platforms \cite{intel_2021}.\\

The above algorithm is first tried with a stride of 1 byte. The latencies are then plotted to obtain the following plots in Fig.\ref{Fig:blockSize}. From this, we can see that for every 64 bytes, there is a peak in the time taken. This tells us that every 64$^{th}$ byte that is accessed, there is a miss in the cache and, the required cache block has to be retrieved from memory - giving us the cache block size.
\begin{figure}[H]
    \centering
    \setlength\tabcolsep{1pt}
    \begin{tabular}{cc}
        \def\svgwidth{0.5\columnwidth}
        \input{blockSize_1B.pdf_tex}&
        \def\svgwidth{0.5\columnwidth}
        \input{blockSize.pdf_tex}
    \end{tabular}
    \caption{Calculating L1-D\$ block size}
    \label{Fig:blockSize}
\end{figure}

Another way to confirm the same is by taking different strides across the array, and plotting the access latencies for each memory access. As shown in Fig.\ref{Fig:blockSize}, we can see that every 2$^{nd}$ 32 byte stride takes a high latency, every 4$^{th}$ 16 byte stride takes a high latency, and almost every 64 byte stride takes a high latency. Thus, we again get a block size of 64 bytes.

\subsection{Measuring Associativity}
On obtaining the block size using above technique, we can reverse-engineer the associativity of the cache by repeatedly accessing blocks that we expect to belong to the same set.


%Beginning References. Don't add any text beyond this.
%------------------------------------------

\newpage %sending References to the last page

\bibliography{paper}
\bibliographystyle{acm}
\end{document}
