\documentclass[12pt,a4paper,english]{paper} 
\usepackage{fontspec}
\usepackage[utf8]{inputenc}
\usepackage[margin=2cm]{geometry}
\usepackage{multirow}
\usepackage{hyperref}
\usepackage{graphicx}
\usepackage{bm}
\usepackage[usenames,dvipsnames]{color}
\usepackage{booktabs}
\usepackage{fancyhdr}
\usepackage[most]{tcolorbox}
\usepackage{changepage}
\usepackage[square,sort,comma,numbers]{natbib}
\usepackage{amsmath}
\usepackage{amssymb}
\usepackage{eucal}
\usepackage[]{minted}
\usepackage{latexsym}
\usepackage{indentfirst}
\usepackage[ruled,vlined]{algorithm2e}
\usepackage[english]{babel}
\usepackage[autostyle, english = american]{csquotes}
\MakeOuterQuote{"}

\setmonofont[Scale=0.75]{Hack NF}

\def \courseNumber {CS6600}
\def \courseName {Computer Architecture}
\def \assignmentName {Reverse Engineering L1 Cache}
\def \myName {Akilesh Kannan}
\def \rollNumber {EE18B122}

\setlength{\headheight}{14pt}

\pagestyle{fancy}
\fancyhf{}
\rhead{\assignmentName}
\lhead{\courseNumber: \courseName}
\cfoot{\thepage}

% \linespread{1.2}

\renewcommand{\familydefault}{\sfdefault} %command to change font to sans-serif

\definecolor{blue(ryb)}{rgb}{0.01, 0.28, 1.0}
\definecolor{green(ryb)}{rgb}{0.28, 1.0, 0.01}
\definecolor{red(ryb)}{rgb}{1.0, 0.01, 0.28}
\definecolor{black(ryb)}{rgb}{0, 0, 0}
\definecolor{gray(ryb)}{rgb}{0.75, 0.75, 0.75}
\definecolor{orange}{RGB}{255,155,0}
\definecolor{formalblue}{rgb}{0.95,0.95,1}
\definecolor{formalred}{rgb}{1,0.95,0.95}

\newenvironment{colorboxed}[4][gray]{
\begin{tcolorbox}[colback=#1!3!white,colframe=#1(ryb)!50!black,title=\textbf{#2 #3},#4]
}{
\end{tcolorbox}
}

\newenvironment{warning}{%
  \def\FrameCommand{%
    \hspace{1pt}%
    {\color{red}\vrule width 2pt}%
    {\color{formalred}\vrule width 4pt}%
    \colorbox{formalred}%
  }%
  \MakeFramed{\advance\hsize-\width\FrameRestore}%
  \noindent\hspace{-4.55pt}% disable indenting first paragraph
  \begin{adjustwidth}{7pt}{}%
  \vspace{2pt}\vspace{2pt}%
}
{%
  \vspace{2pt}\end{adjustwidth}\endMakeFramed%
}

\newenvironment{results}{%
  \def\FrameCommand{%
    \hspace{1pt}%
    {\color{blue}\vrule width 2pt}%
    {\color{formalblue}\vrule width 4pt}%
    \colorbox{formalblue}%
  }%
  \MakeFramed{\advance\hsize-\width\FrameRestore}%
  \noindent\hspace{-4.55pt}% disable indenting first paragraph
  \begin{adjustwidth}{7pt}{}%
  \vspace{2pt}\vspace{2pt}%
}
{%
  \vspace{2pt}\end{adjustwidth}\endMakeFramed%
}

\begin{document} 
\thispagestyle{empty}
\vspace{-4.5cm}

\hspace*{-\parindent}
\begin{minipage}{0.65\textwidth}
\fontsize{22pt}{10pt}\selectfont\textbf{\assignmentName}\\[1mm]
\Large
\textit{\courseNumber: \courseName}\\[5mm]
\Large \myName \\
\normalsize \rollNumber \\
\end{minipage}\hfill% push everything to the right
\raisebox{-13mm}{\includegraphics[scale=.28]{logo.pdf}}

\hrule \hrule
\medskip


\begin{warning}
    This assignment was worked on individually.\\
    The programs and tools used can be found here - \url{https://github.com/aklsh/CacheEval}
\end{warning}

\section{System Details}
\begin{table}[H]
    \centering
    \resizebox{0.5\textwidth}{!}{%
    \begin{tabular}{|c|c|c|}
        \hline
         \textbf{Parameter} & \multicolumn{2}{c|}{\textbf{Value}} \\ \cline{2-3} 
            & \textit{Predicted} & \textit{Expected \footnotemark} \\ \hline
        {Associativity}        &         8          &        8            \\ \hline
        {Block Size}           &       64 Bytes     &     64 Bytes        \\ \hline
        {Processor Name}       &         -          &  AMD Ryzen 7 4800HS \\ \hline
        {ISA}                  &         -          &     x86\_64         \\ \hline
        {L1 D\$ size}          &         -          &      32 KB          \\ \hline
    \end{tabular}
    }
    \caption{Summary Table}
\end{table}
\footnotetext{On GNU/Linux OSes, it is obtained using \texttt{grep . /sys/devices/system/cpu/cpu0/cache/index*/*}}

\section{Approach}
\begin{results}
    \textsl{Misses in the cache lead to increased latency in memory access.}
\end{results}

\subsection{Measuring Block Size}
\begin{algorithm}[H]
\SetAlgoLined
\KwResult{Block Size}
 \SetKwInput{input}{Input}
 \input{stride values}
 \ForEach{stride}{
 \SetKw{init}{initialise} \init cacheBlock[64M], sum $=$ 0, startTime $=$ 0, endTime $=$ 0\; 
 \SetKw{clear}{clear} \clear cache\;
 \For{i $\leftarrow$ 1 \KwTo 64M}{
  startTime $\leftarrow$ RDTSC\;
    \SetKw{access}{access}\access cacheBlock[i]\;
  endTime $\leftarrow$ RDTSC\;
  \SetKw{print}{print}\print endTime $-$ startTime\;
  i $\leftarrow$ i $+$ stride\;
 }
}
    \SetKw{plot}{plot} \plot data\;
 \caption{Calculating Block Size of L1 D\$}
    \label{Alg1}
\end{algorithm}

The \texttt{RDTSC} instruction returns the data in the Time-Stamp Counter in x86 platforms \cite{intel_2021}.\\

Algorithm \ref{Alg1} is first tried with a stride of 1 byte. The latencies are then plotted to obtain the plots in Fig.\ref{Fig:blockSize}. From here, we can see that for every 64 bytes, there is a peak in the latency. This tells us that every 64$^{th}$ byte that is accessed, there is a miss in the cache and, the required cache block has to be retrieved from memory - giving us the cache block size.

Another way to confirm the same is by taking different strides across the array, and plotting the access latencies for each memory access. As shown in Fig.\ref{Fig:blockSize}, we can see that every 2$^{nd}$ 32 byte stride takes a high latency, every 4$^{th}$ 16 byte stride takes a high latency, and almost every 64 byte stride takes a high latency. Thus, we again get a block size of 64 bytes.

\begin{figure}[H]
    \centering
    \setlength\tabcolsep{1pt}
    \begin{tabular}{cc}
        \def\svgwidth{0.5\columnwidth}
        \input{blockSize_1B.pdf_tex}&
        \def\svgwidth{0.5\columnwidth}
        \input{blockSize.pdf_tex}
    \end{tabular}
    \caption{Calculating L1-D\$ block size}
    \label{Fig:blockSize}
\end{figure}

\subsection{Measuring Associativity}
We can reverse-engineer the associativity of the cache by repeatedly accessing blocks that we expect to belong to the same set. On encountering a higher latency for one block, we can infer that it was replaced and hence took more time, giving us the associativity of the cache.

\begin{algorithm}[H]
\SetAlgoLined
\KwResult{Associativity}
 \KwIn{ASSOC, where ASSOC is the assumed value of associativity}
 \SetKw{init}{initialise} \init cacheBlock[64M], i $=$ 0, startTime $=$ 0, endTime $=$ 0\; 
 \SetKw{clear}{clear} \clear cache\;
    \While{i $\leq$ ASSOC}{
  startTime $\leftarrow$ RDTSC\;
    \SetKw{access}{access}\access cacheBlock[i $*$ CACHE\_SIZE]\;
  endTime $\leftarrow$ RDTSC\;
  \SetKw{print}{print}\print endTime $-$ startTime\;
  i $\leftarrow$ i $+$ 1\; 
    }
    \SetKw{reset}{reset} \reset i $=$ 0\;
    \While{i $\leq$ ASSOC}{
  startTime $\leftarrow$ RDTSC\;
    \SetKw{access}{access}\access cacheBlock[i $*$ CACHE\_SIZE]\;
  endTime $\leftarrow$ RDTSC\;
  \SetKw{print}{print}\print endTime $-$ startTime\;
  i $\leftarrow$ i $+$ 1\; 
    }
 \caption{Calculating Associativity of L1 D\$}
    \label{Alg2}
\end{algorithm}

Algorithm \ref{Alg2} was tried with the following \texttt{ASSOC} values - 2, 4, 8, 16, as most modern processors fall into one of these categories and the following outputs were got:

\begin{colorboxed}{Output: }{\texttt{\textbf{ASSOC = 2}}}{breakable}
\begin{verbatim}
========== ASSOCIATIVITY CALCULATION ==========
---------START Populating---------
Byte 0:         1160
Byte 32768:     12412
Byte 65536:     7221
---------FINISH Populating---------
---------START New Access---------
Byte 0:         203
Byte 32768:     203
Byte 65536:     203
---------FINISH New Access---------
\end{verbatim}
\end{colorboxed}

\begin{colorboxed}{Output: }{\texttt{\textbf{ASSOC = 4}}}{breakable}
\begin{verbatim}
========== ASSOCIATIVITY CALCULATION ==========
---------START Populating---------
Byte 0:         609
Byte 32768:     15457
Byte 65536:     5597
Byte 98304:     11948
Byte 131072:    7772
---------FINISH Populating---------
---------START New Access---------
Byte 0:         203
Byte 32768:     203
Byte 65536:     203
Byte 98304:     203
Byte 131072:    203
---------FINISH New Access---------
\end{verbatim}
\end{colorboxed}


\begin{colorboxed}{Output: }{\texttt{\textbf{ASSOC = 8}}}{breakable}
\begin{verbatim}
========== ASSOCIATIVITY CALCULATION ==========
---------START Populating---------
Byte 0:         1218
Byte 32768:     14935
Byte 65536:     6699
Byte 98304:     5655
Byte 131072:    6757
Byte 163840:    4698
Byte 196608:    5684
Byte 229376:    4524
Byte 262144:    22098
---------FINISH Populating---------
---------START New Access---------
Byte 0:         580
Byte 32768:     203
Byte 65536:     203
Byte 98304:     203
Byte 131072:    232
Byte 163840:    203
Byte 196608:    203
Byte 229376:    203
Byte 262144:    203
---------FINISH New Access---------
\end{verbatim}
\end{colorboxed}

\begin{colorboxed}{Output: }{\texttt{\textbf{ASSOC = 16}}}{breakable}
\begin{verbatim}
========== ASSOCIATIVITY CALCULATION ==========
---------START Populating---------
Byte 0:         1073
Byte 32768:     13746
Byte 65536:     11571
Byte 98304:     5945
Byte 131072:    5742
Byte 163840:    5742
Byte 196608:    6612
Byte 229376:    5771
Byte 262144:    75951
Byte 294912:    5742
Byte 327680:    11078
Byte 360448:    6032
Byte 393216:    5945
Byte 425984:    5916
Byte 458752:    5858
Byte 491520:    6612
Byte 524288:    5655
---------FINISH Populating---------
---------START New Access---------
Byte 0:         493
Byte 32768:     203
Byte 65536:     203
Byte 98304:     203
Byte 131072:    203
Byte 163840:    203
Byte 196608:    203
Byte 229376:    203
Byte 262144:    203
Byte 294912:    203
Byte 327680:    203
Byte 360448:    203
Byte 393216:    203
Byte 425984:    203
Byte 458752:    203
Byte 491520:    203
Byte 524288:    174
---------FINISH New Access---------
\end{verbatim}
\end{colorboxed}

From the above results, we can see that for \texttt{ASSOC} of 2 or 4, any extra block doesn't cause another to be evicted - all access times remain same. But for \texttt{ASSOC} of 8, we see an increase in the time taken to access block 0. This can be inferred as the time taken to bring it from L2, as it was evicted in the previous round. We see a similar pattern for \texttt{ASSOC} of 16 - some blocks take higher latency than others.

%Beginning References. Don't add any text beyond this.
%------------------------------------------

\newpage %sending References to the last page

\bibliography{paper}
\bibliographystyle{acm}
\end{document}
